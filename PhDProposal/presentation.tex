%% This Beamer template is based on the one found here: https://github.com/sanhacheong/stanford-beamer-presentation, and edited to be used for Stanford ARM Lab

\documentclass[10pt]{beamer}
\usetheme{Frankfurt}
%\usecolortheme{beaver}
%\mode<presentation>{}
\usepackage{amssymb,amsmath,amsthm,enumerate}
\usepackage[utf8]{inputenc}
\usepackage{array}
\usepackage[parfill]{parskip}
\usepackage{graphicx}
\usepackage{caption}
\usepackage{subcaption}
\usepackage{bm}
\usepackage{amsfonts,amscd}
\usepackage[]{units}
\usepackage{listings}
\usepackage{multicol}
\usepackage{multirow}
\usepackage{tcolorbox}
\usepackage{physics}
\usepackage{xcolor}
\usepackage{hyperref}

% Enable colored hyperlinks
\hypersetup{colorlinks=true}

% The following three lines are for crossmarks & checkmarks
\usepackage{pifont}% http://ctan.org/pkg/pifont
\newcommand{\cmark}{\ding{51}}%
\newcommand{\xmark}{\ding{55}}%

% Numbered captions of tables, pictures, etc.
\setbeamertemplate{caption}[numbered]

%\usepackage[superscript,biblabel]{cite}
\usepackage{algorithm2e}
\renewcommand{\thealgocf}{}

% Bibliography settings
\usepackage[style=apa]{biblatex}
\setbeamertemplate{bibliography item}{\insertbiblabel}
\addbibresource{references/rq1.bib}
\addbibresource{references/rq2.bib}
\addbibresource{references/rq3.bib}
\addbibresource{references/rq4.bib}
\addbibresource{references/rq5.bib}
%\addbibresource{references/references.bib}
% Glossary entries
\usepackage[acronym]{glossaries}
\newacronym{ML}{ML}{machine learning}
\newacronym{HRI}{HRI}{human-robot interactions}
\newacronym{RNN}{RNN}{Recurrent Neural Network}
\newacronym{LSTM}{LSTM}{Long Short-Term Memory}


\theoremstyle{remark}
\newtheorem*{remark}{Remark}
\theoremstyle{definition}

\newcommand{\empy}[1]{{\color{darkorange}\emph{#1}}}
\newcommand{\empr}[1]{{\color{cardinalred}\emph{#1}}}
\newcommand{\examplebox}[2]{
\begin{tcolorbox}[colframe=darkcardinal,colback=boxgray,title=#1]
#2
\end{tcolorbox}}

%\usetheme{default} 
%\input{./style_files_stanford/my_beamer_defs.sty}

\logo{\includegraphics[height=0.4in]{images/univalle.png}}

% commands to relax beamer and subfig conflicts
% see here: https://tex.stackexchange.com/questions/426088/texlive-pretest-2018-beamer-and-subfig-collide
\makeatletter
\let\@@magyar@captionfix\relax
\makeatother

\title[Research proposal]{Characterizing and understanding security risks through Fuzzing Secure-Aware Mutation Testing of RESTFul-API services}
%\subtitle{Subtitle Of Presentation}

%\beamertemplatenavigationsymbolsempty
\AtBeginSection[]{
  \begin{frame}
  \vfill
  \centering
  \begin{beamercolorbox}[sep=8pt,center,shadow=true,rounded=true]{title}
    \usebeamerfont{title}\insertsectionhead\par%
  \end{beamercolorbox}
  \vfill
  \end{frame}
}

\DeclareUnicodeCharacter{0301}{\'{e}}

\author[Carlos Delgado]{
    \large{Carlos Andres Delgado Saavedra}\\
    \footnotesize \href{mailto:carlos.andres.delgado@correounivalle.edu.co}{carlos.andres.delgado@correounivalle.edu.co} 
    %Advisors:\\
    %Jesus A. Aranda, Universidad del Valle \\
    %James Ortiz, Universite de Namur
    }
\begin{document}




\institute{
	\vskip 5pt
	\begin{figure}
		\centering
		\begin{subfigure}[t]{0.33\textwidth}
			\centering
			\includegraphics[height=0.4in]{images/univalle.png}
		\end{subfigure}%
		\begin{subfigure}[t]{0.33\textwidth}
			\centering
			\includegraphics[height=0.4in]{images/logounamur.png}
		\end{subfigure}%
		~ 
		\begin{subfigure}[t]{0.33\textwidth}
			\centering
			\includegraphics[height=0.4in]{images/avispalogo.png}
		\end{subfigure}
	\end{figure}
	\vskip 5pt
	%Escuela de Ingenieria de Sistemas y Computacion\\
	%Universidad del Valle\\
	Colombia
	\vskip 3pt
}

\date{February 2nd, 2024}


\begin{frame}[plain]\maketitle\end{frame}


\setbeamertemplate{itemize items}[default]
\setbeamertemplate{itemize subitem}[circle]

\begin{frame}
	\frametitle{Overview} % Table of contents slide, comment this block out to remove it
	\tableofcontents % Throughout your presentation, if you choose to use \section{} and \subsection{} commands, these will automatically be printed on this slide as an overview of your presentation
\end{frame}


\section{Research Proposal}

\begin{frame}{Problem}

\begin{enumerate}
    \item API-RESTFul is an architectural style for designing web services
    \item RESTFul APIs exchange sensitive information and private date
    \item Top 10 vulnerabilities Application Security Project (OWASP) \url{https://owasp.org/www-project-api-security/}
    \item Coverage of the security tests: penetration and policies
    \item Opportunity for mutation testing
\end{enumerate}
\end{frame}


\begin{frame}{Research question}
¿How to design fuzzed secure-aware mutation operators in the coverage of the vulnerabilities in the configuration of security policies in RESTFul APis?
\end{frame}



\begin{frame}{Objectives}

Develop a collection of security-aware mutation operators designed for safeguarding the configuration of security policies within RESTFul API services.

\end{frame}

\begin{frame}{Specific}

\begin{table}[H]
    \centering
    \begin{tabular}{|p{0.5\textwidth}|p{0.5\textwidth}|}
        \hline
         \textbf{Specific objective} & \textbf{Expected result} \\ \hline
         1.  Identification of the elements of the security policies in API-RESTFul services &  Characteristics of the security policies in API-Restful services  \\  \hline
         2. Describe a set of fuzzed security-aware mutation operators for testing of security policies in API-RESTFul services & Description of the mutation operators according to the elements of security policies in API-Restful services  \\  \hline
         3. Develop the set of security-aware mutation operators for testing in Django Rest and Flask Frameworks in Python    & Source code of the secure-aware mutation operators \\  \hline
         4. Evaluate the proposed security-aware mutation operators in RESTFul API services & Report about the performance of the created operators against tools from the literature.
         \\ \hline
    \end{tabular}
    \caption{Specific objectives and expected results}
    \label{tab:objetivos}
\end{table}
\end{frame}

\section{Literature Review}

\begin{frame}{Strategy}
  \begin{enumerate}
    \item Questions about the current state of art in the configuration security policies of RESTFul APIs.
    \item Window of time from 2000 to 2024. \url{https://doi.org/10.1515/itit-2013-1035}
    \item Emphasis in the last 5 years. \url{https://doi.org/10.1145/3617175}, \url{https://journal.ijresm.com/index.php/ijresm/article/view/970} the rise of the RESTFul APIs.
    \end{enumerate}
\end{frame}

\begin{frame}{Research questions}
	\begin{enumerate}
    \item RQ1: What are the elements of the security configuration policies in the RESTFul API Services?
		\item RQ2: What are the current challenges about the security policies of RESTFul API Services?
    \item RQ3: What are the most common configuration security mistakes of the developers in the building of RESTFul API Services?
    \item RQ4: What are the current testing techniques and tools for the testing of configuration policies of RESTFul API Services based on Python?
    \item RQ5: What experiences have been reported in the literature about the use of mutation testing for the security testing of RESTFul API Services?
	\end{enumerate}
\end{frame}

\nocite{*}

\begin{frame}{RQ1: Elements of security configuration policies}
  %\cite{Siriwardena2020, Subramanian2019-ld, Luo2016, Madden2021-oi, Kellezi2019}
  \begin{enumerate}
    \item Authentication: Methods for the identification of the user.
      \item Authorization: Methods for the access control.
    \item Encryption: Protocol SSL/TLS.
    \item Data masking: Hide sensitive data in logs and responses.
    \item Input validation and sanitization: Prevent injection attacks (SQL, XSS).
    \item Thottling: Number of requests per time.
    \item API Keys: Each user with their own key.
    \item Login level: Detailed and security monitoring.
  \end{enumerate}
\end{frame}

\begin{frame}[allowframebreaks]{RQ1: References}
  \printbibliography[keyword={RQ1}]
\end{frame}
  

\begin{frame}{RQ2: Current challenges}
  %\cite{Kellezi2019}
  \begin{enumerate}
    \item Keep the data integrity in RESTFul API Services is a challenge that changes every day.
    \item Several recent studies have identified security gaps in many of them.
    \item One of the most problems about software vulnerabilities is the configuration security policies of RESTFul APIs
    \item Testing methods and tools are not enough to cover all the vulnerabilities.
  \end{enumerate}
\end{frame}

\begin{frame}[allowframebreaks]{RQ2: References}
  \printbibliography[keyword={RQ2}]
  \end{frame}

  \begin{frame}{RQ3: Common configuration mistakes}
    \begin{enumerate}
      \item Lack of input validation.
      \item Insecure deserialization.
      \item Lack of proper authentication and authorization.
      \item Insecure direct object references.
      \item Lack of proper logging and monitoring.
        \item Insecure communication with untrusted components.
    \end{enumerate}
  \end{frame}

  \begin{frame}[allowframebreaks]{RQ3: References}
    \printbibliography[keyword={RQ3}]
  \end{frame}

  \begin{frame}{RQ4: Testing techniques and tools}
    Penetration testing, vulnerability assessment, and network scanning. 
    \begin{enumerate}
      \item OWASP ZAP: Penetration testing.
      \item Postman: API testing.
      \item Burp Suite: Penetration testing.
      \item Nessus: Vulnerability assessment.
      \item Nmap: Network scanning.
      \item Metasploit: Penetration testing.
    \end{enumerate}
    Techniques: Fuzzing, black box, statistical.
  \end{frame}

  \begin{frame}[allowframebreaks]{RQ4: References}
    \printbibliography[keyword={RQ4}]
  \end{frame}

  \begin{frame}{RQ5: Mutation testing in security of RESTFul API Services}
    \begin{enumerate}
      \item Mutation testing has proven to be a strategy for evaluating the security of applications.
      \item The literature suggests an emphasis in data integrity.
      \item Different strategies for the mutation testing: using artificial intelligence, black box testing, penetration testing, validation of data integrity and statistical methods.
    \end{enumerate}
  \end{frame}

  \begin{frame}[allowframebreaks]{RQ5: References}
   \printbibliography[keyword={RQ5}]
  \end{frame}

\subsection{Next tasks}

\begin{frame}{Tasks}
  \begin{enumerate}
    \item Finish the literature review: Categories and subcategories. Article of the review of the state of the art.
    \item Adjust the proposal according to this guidelines.
    \item Defense of the proposal.
  \end{enumerate}
\end{frame}

\subsection{Challenges}

\begin{frame}{Challenges}
\begin{enumerate}
    \item RESTFul APIs handle sensitive information that needs to be protected, software testing evaluates how they are handled, but because vulnerabilities are constantly being discovered, there is an opportunity for improvement in this area.
    \item Mutation testing has proven to be a strategy for evaluating the security of applications, there has been a lot of work done related to specific applications in languages such as Java and Python, there is an opportunity to contribute to the development of RESTFul API.
    \item Security is a challenge for software development today, and several recent studies have identified security gaps in many of them, which could be studied to provide a framework for the development of tools to assess data security and generate recommendations for improvement.
\end{enumerate}
\end{frame}


\section{Work plan}

\begin{frame}{Contribution selection}
Working plan: Following the snowball methodology
\begin{itemize}
    \item Review of vulnerabilities in RESTFul APIs: Survey in the interception between mutation testing and security evaluation in Restful-API.
    \item Description of the mutation operators
    \item Prototype implementation and testing
\end{itemize}
Total: 3 years.
\end{frame}


\begin{frame}[allowframebreaks]
\frametitle{References}
  \printbibliography
\end{frame}

\end{document}
