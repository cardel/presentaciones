%% This Beamer template is based on the one found here: https://github.com/sanhacheong/stanford-beamer-presentation, and edited to be used for Stanford ARM Lab

\documentclass[10pt]{beamer}
\usetheme{Frankfurt}
%\usecolortheme{beaver}
%\mode<presentation>{}
\usepackage{amssymb,amsmath,amsthm,enumerate}
\usepackage[utf8]{inputenc}
\usepackage{array}
\usepackage[parfill]{parskip}
\usepackage{graphicx}
\usepackage{caption}
\usepackage{subcaption}
\usepackage{bm}
\usepackage{amsfonts,amscd}
\usepackage[]{units}
\usepackage{listings}
\usepackage{multicol}
\usepackage{multirow}
\usepackage{tcolorbox}
\usepackage{physics}
\usepackage{xcolor}
\usepackage{hyperref}

% Enable colored hyperlinks
\hypersetup{colorlinks=true}

% The following three lines are for crossmarks & checkmarks
\usepackage{pifont}% http://ctan.org/pkg/pifont
\newcommand{\cmark}{\ding{51}}%
\newcommand{\xmark}{\ding{55}}%

% Numbered captions of tables, pictures, etc.
\setbeamertemplate{caption}[numbered]

%\usepackage[superscript,biblabel]{cite}
\usepackage{algorithm2e}
\renewcommand{\thealgocf}{}

% Bibliography settings
\usepackage[style=apa]{biblatex}
\setbeamertemplate{bibliography item}{\insertbiblabel}
\addbibresource{references/rq1.bib}
\addbibresource{references/rq2.bib}
\addbibresource{references/rq3.bib}
\addbibresource{references/rq4.bib}
\addbibresource{references/rq5.bib}
\addbibresource{references/methodology.bib}
%\addbibresource{references/references.bib}
% Glossary entries
\usepackage[acronym]{glossaries}
\newacronym{ML}{ML}{machine learning}
\newacronym{HRI}{HRI}{human-robot interactions}
\newacronym{RNN}{RNN}{Recurrent Neural Network}
\newacronym{LSTM}{LSTM}{Long Short-Term Memory}


\theoremstyle{remark}
\newtheorem*{remark}{Remark}
\theoremstyle{definition}

\newcommand{\empy}[1]{{\color{darkorange}\emph{#1}}}
\newcommand{\empr}[1]{{\color{cardinalred}\emph{#1}}}
\newcommand{\examplebox}[2]{
\begin{tcolorbox}[colframe=darkcardinal,colback=boxgray,title=#1]
#2
\end{tcolorbox}}

%\usetheme{default} 
%\input{./style_files_stanford/my_beamer_defs.sty}

\logo{\includegraphics[height=0.4in]{images/univalle.png}}

% commands to relax beamer and subfig conflicts
% see here: https://tex.stackexchange.com/questions/426088/texlive-pretest-2018-beamer-and-subfig-collide
\makeatletter
\let\@@magyar@captionfix\relax
\makeatother

\title[Research proposal]{Characterizing and understanding security risks through Security-Aware Mutation Testing of security configuration in RESTful APIs}
%\subtitle{Subtitle Of Presentation}

%\beamertemplatenavigationsymbolsempty
\AtBeginSection[]{
  \begin{frame}
  \vfill
  \centering
  \begin{beamercolorbox}[sep=8pt,center,shadow=true,rounded=true]{title}
    \usebeamerfont{title}\insertsectionhead\par%
  \end{beamercolorbox}
  \vfill
  \end{frame}
}

\DeclareUnicodeCharacter{0301}{\'{e}}

\author[Carlos Delgado]{
    \large{Carlos Andres Delgado Saavedra}\\
    \footnotesize \href{mailto:carlos.andres.delgado@correounivalle.edu.co}{carlos.andres.delgado@correounivalle.edu.co} 
    %Advisors:\\
    %Jesus A. Aranda, Universidad del Valle \\
    %James Ortiz, Universite de Namur
    }

\usepackage{tikz}
\usetikzlibrary{shapes.geometric, arrows, positioning}


\begin{document}

\institute{
	\vskip 5pt
	\begin{figure}
		\centering
		\begin{subfigure}[t]{0.33\textwidth}
			\centering
			\includegraphics[height=0.4in]{images/univalle.png}
		\end{subfigure}%
		\begin{subfigure}[t]{0.33\textwidth}
			\centering
			\includegraphics[height=0.4in]{images/logounamur.png}
		\end{subfigure}%
		~ 
		\begin{subfigure}[t]{0.33\textwidth}
			\centering
			\includegraphics[height=0.4in]{images/avispalogo.png}
		\end{subfigure}
	\end{figure}
	\vskip 5pt
	%Escuela de Ingenieria de Sistemas y Computacion\\
	%Universidad del Valle\\
	%Colombia
	\vskip 3pt
}

\date{August, 2024}


\begin{frame}[plain]\maketitle\end{frame}


\setbeamertemplate{itemize items}[default]
\setbeamertemplate{itemize subitem}[circle]

\begin{frame}
	\frametitle{Overview} % Table of contents slide, comment this block out to remove it
	\tableofcontents % Throughout your presentation, if you choose to use \section{} and \subsection{} commands, these will automatically be printed on this slide as an overview of your presentation
\end{frame}


\section{Research Proposal}

\begin{frame}
    \frametitle{Context}
    \begin{itemize}
        \item RESTful API: An architectural style for designing web services.
        \begin{itemize}
            \item Uses HTTP requests to access resources.
            \item Offers flexibility and scalability for system communication.
        \end{itemize}
        \item Security challenges in RESTful APIs:
        \begin{itemize}
            \item Exchange of sensitive data (passwords, credit card numbers, personal information).
            \item Vulnerabilities due to lack of authentication and authorization.
        \end{itemize}
        \item Modern security practices:
        \begin{itemize}
            \item Encrypting communication.
            \item Requiring authentication.
            \item Input validation.
            \item Restricting resource access.
        \end{itemize}
    \end{itemize}
\end{frame}

\begin{frame}
    \frametitle{The Problem}
    \begin{itemize}
        \item RESTful APIs often handle sensitive and private data.
        \item Critical security mechanisms:
        \begin{itemize}
            \item Authorization and access policies.
            \item Access restrictions and encryption.
        \end{itemize}
        \item OWASP 2023 reports an increase in API security risks:
        \begin{itemize}
            \item Authorization lacking.
            \item Uncontrolled resource consumption.
            \item Security misconfiguration.
            \item Unauthorized data access.
        \end{itemize}
        \item Companies must invest in:
        \begin{itemize}
            \item Updating applications and security policies.
            \item Monitoring data exchange.
            \item Implementing encryption protocols (HTTPS/TLS).
            \item Authorization mechanisms (OAuth).
            \item Access restrictions (CORS).
        \end{itemize}
    \end{itemize}
\end{frame}

\begin{frame}[allowframebreaks]
    \frametitle{The Importance of Software Testing}
    \begin{itemize}
        \item Growing importance of software testing in detecting vulnerabilities:
        \begin{itemize}
            \item Early identification and fixing of vulnerabilities.
            \item Prevent exploitation by attackers.
        \end{itemize}
        \item Common vulnerabilities in RESTful APIs:
        \begin{itemize}
            \item Broken object-level authorization.
            \item Broken user authentication.
            \item Excessive data exposure.
        \end{itemize}
        \item Other security risks:
        \begin{itemize}
            \item Injection attacks (malicious code).
            \item Rate limiting attacks (API overload).
            \item Denial-of-service attacks.
        \end{itemize}
    \end{itemize}
\end{frame}

\begin{frame}[allowframebreaks]
    \frametitle{Role of Mutation Testing}
    \begin{itemize}
        \item Mutation testing: A tool to evaluate security test capabilities.
        \begin{itemize}
            \item Creates new scenarios by mutating code.
            \item Helps identify potential new vulnerabilities.
        \end{itemize}
        \item Benefits of mutation testing:
        \begin{itemize}
            \item Detects unexpected vulnerabilities.
            \item Simulates risk situations exploited by attackers.
        \end{itemize}
        \item Need for security-aware mutation operators:
        \begin{itemize}
            \item Provides a framework for security tests.
            \item Evaluates the quality of security tests performed by developers.
        \end{itemize}
    \end{itemize}
\end{frame}

\begin{frame}{Research question}
How can security-aware mutation operators be designed to improve the coverage of security testing for vulnerabilities in the configuration of security policies in RESTful APIs?
\end{frame}


\begin{frame}{Objectives}

Develop a collection of security-aware mutation operators designed for the evaluation of the configuration of security policies files within RESTful APIs.

\end{frame}

\begin{frame}{Specific}

\begin{table}[H]
    \centering
    \begin{tabular}{|p{0.5\textwidth}|p{0.5\textwidth}|}
        \hline
         \textbf{Specific objective} & \textbf{Expected result} \\ \hline
         1. Identification of the elements of the security policies in RESTful APIs  &  Characteristics of the security policies in RESTful API, related to exchanging of data   \\  \hline
         2. Describe a set of code-based security-aware mutation operators for testing of security policies   files in RESTful APIs  & Description of the mutation operators, introducing some misconfiguration security policies in the exchanging of data in RESTful APIs\\  \hline
         3. Develop the set of security-aware mutation operators for security configuration files  & Description of the operators to be applied in security configuration files \\  \hline
         4. Evaluate the proposed security-aware mutation operators in the coverage of the security tests  & Report about the performance of the created operators against tools from the literature.
         \\ \hline
    \end{tabular}
    \caption{Specific objectives and expected results}
    \label{tab:objetivos}
\end{table}
\end{frame}

\section{Methodology}

\subsection{Description of the methodology}

\begin{frame}
    \frametitle{Introduction}
    \begin{itemize}
        \item Review of vulnerabilities in RESTful APIs
        \item Description of mutation operators
        \item Prototype implementation and testing
    \end{itemize}
\end{frame}

\begin{frame}
    \frametitle{Review of Vulnerabilities in RESTful APIs}
    \begin{itemize}
        \item Research methodology: Snowballing \cite{Chaim2008}
        \item Initial focus: Recent surveys on mutation testing \cite{Papadakis2019}, testing challenges for RESTful APIs \cite{Ehsan2022}, and software security testing \cite{Golmohammadi2023}.
        \item Identification of common vulnerabilities and mitigation strategies.
        \item Focus on OWASP 2023 top 10 vulnerabilities.
    \end{itemize}
\end{frame}

\begin{frame}
    \frametitle{OWASP Top 10 Vulnerabilities for 2023}
    \begin{enumerate}
        \item Broken object-level authorization
        \item Broken authentication
        \item Unrestricted resource consumption
        \item Broken authorization at the role level
        \item Unrestricted access to sensitive business flows
        \item Server-side request forgery (SSRF)
        \item Security misconfiguration
        \item Inadequate inventory management
        \item Insecure API consumption
    \end{enumerate}
\end{frame}

\begin{frame}
    \frametitle{Objective of the Vulnerability Review}
    \begin{itemize}
        \item Explore characteristics of common vulnerabilities.
        \item Analyze how these vulnerabilities are handled in the software development process.
        \item Identify strategies used to mitigate vulnerabilities.
        \item Determine mutation operators to implement in the prototype.
    \end{itemize}
\end{frame}

\begin{frame}
    \frametitle{Description of Mutation Operators}
    \begin{itemize}
        \item Define strategy to introduce vulnerabilities into source code.
        \item Variations of mutation operators to produce vulnerability effects.
        \item Analyze possible redundant mutants produced by the mutation operators.
    \end{itemize}
\end{frame}

\begin{frame}
    \frametitle{Mutation Operators: Implementation Details}
    \begin{itemize}
        \item Focus on modifying:
        \begin{itemize}
            \item Configuration files of the RESTful API
            \item Source code of the API
            \item Test cases
        \end{itemize}
        \item Goal: Introduce vulnerabilities to analyze and test mitigation strategies.
    \end{itemize}
\end{frame}

\begin{frame}
    \frametitle{Prototype Implementation and Testing}
    \begin{itemize}
        \item Approach: Test-Driven Development (TDD) \cite{williams2003test}
        \item Generate test cases from mutation operator descriptions.
        \item Validate the effect of mutation operators in introducing and identifying vulnerabilities.
    \end{itemize}
\end{frame}

\begin{frame}
    \frametitle{Expected Outcomes}
    \begin{itemize}
        \item Successful identification of vulnerabilities through mutation testing.
        \item Effective mitigation strategies for each identified vulnerability.
        \item A comprehensive list of mutation operators applicable to RESTful API security testing.
    \end{itemize}
\end{frame}

\section{Phases of the Project}
\begin{frame}
    \frametitle{Phases of the Project}
    This project defines four phases to approach the objectives:
    \begin{enumerate}
        \item Systematic review of the literature \cite{Kitchenham2002}.
        \item Design of the security-aware mutation operators for RESTful API services \cite{Peffers2007}.
        \item Development of the security-aware mutation operators using TDD methodology.
        \item Evaluation of the mutation operators using metrics \cite{Ahmed2010}.
    \end{enumerate}
\end{frame}

\section{Systematic Review of the Literature}
\begin{frame}
    \frametitle{Systematic Review of the Literature}
    \begin{itemize}
        \item Conduct a systematic review to identify existing security-aware mutation operators.
        \item Steps to follow:
        \begin{enumerate}
            \item Developing a research question.
            \item Identifying relevant databases.
            \item Defining search terms.
            \item Selection criteria.
            \item Data extraction and analysis.
        \end{enumerate}
    \end{itemize}
\end{frame}

\subsection{Research Methodology}

\begin{frame}
    \frametitle{Research Questions}
    Key questions guiding the literature review:
    \begin{enumerate}
        \item What are the existing mutation operators for testing the security of RESTful APIs?
        \item How effective are these mutation operators in detecting security vulnerabilities?
        \item What are the limitations of current mutation operators?
        \item What elements define vulnerabilities in RESTful API services?
        \item How are these vulnerabilities handled in development?
        \item Strategies for mitigating vulnerabilities?
        \item Common security misconfigurations?
    \end{enumerate}
\end{frame}

\begin{frame}
    \frametitle{Design of the Security-aware Mutation Operators}
    The design phase focuses on defining and specifying mutation operators based on identified vulnerabilities.
    \begin{itemize}
        \item Identification of vulnerability elements.
        \item Specification of mutation operators.
        \item Description of mutation application.
        \item Determination of testing elements.
        \item Analysis of coverage and redundancy.
        \item Evaluation of operator effectiveness.
        \item Refinement and iteration.
    \end{itemize}
\end{frame}

\begin{frame}
    \frametitle{Development of the Security-aware Mutation Operators}
    \begin{itemize}
        \item TDD methodology to ensure desired effects.
        \item Steps in the development phase:
        \begin{enumerate}
            \item Selection of case studies using Python frameworks.
            \item Coding mutation operators using tools like MutPy and MutMut.
            \item Analyzing coverage and redundancy metrics.
            \item Evaluating operator effectiveness.
            \item Refactoring code post-test.
        \end{enumerate}
    \end{itemize}
\end{frame}

\begin{frame}
    \frametitle{Evaluation of the Security-aware Mutation Operators}
    The evaluation phase measures the effectiveness of mutation operators using key metrics.
    \begin{itemize}
        \item \textbf{Benchmark Selection:} Choosing RESTful APIs with known vulnerabilities.
        \item \textbf{Mutation Operator Application:} Generating mutant APIs using designed operators.
        \item \textbf{Test Execution:} Running test cases against original and mutant APIs.
        \item \textbf{Evaluation and Analysis:} Using metrics like mutation coverage, fault detection rate, and false positive rate.
    \end{itemize}
\end{frame}

\begin{frame}[allowframebreaks]
\frametitle{References}
  \printbibliography
\end{frame}

\end{document}
